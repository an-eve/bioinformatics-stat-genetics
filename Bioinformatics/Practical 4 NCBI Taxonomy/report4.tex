\documentclass[12 pt,a4paper]{article}
\usepackage{listings}
\usepackage{mathptmx}
\usepackage{savetrees}
\usepackage{minted}
\title{Practical 4: NCBI Taxonomy}
\author{Anna Putina and Marine Mazeau}
\date{10/10/2023, submission deadline 16/10/2023}
\begin{document}
\maketitle
\begin{enumerate}
\item
\setminted{autogobble=true, breaklines=true}
\begin{minted}{python}
import networkx as nx
import pandas as pd
import matplotlib.pyplot as plt

# Read the data
file_path = "nodes.dmp"
df = pd.read_csv(file_path, delimiter = r"\s+\|\s+", usecols=[0, 1, 2], engine = "python", header=None, names = ["child", "parent", "rank"]) 
print('Test small tree: \n', df.head())

# Add ranks
attrs = dict(zip(df['child'], df['rank']))

# Create the graph
G = nx.from_pandas_edgelist(df, source='parent', target='child', create_using=nx.DiGraph())
nx.set_node_attributes(G, attrs, name="rank")

# Check self-loops and remove these edges
print('Nodes with self-loops: ', list(nx.nodes_with_selfloops(G)))
G.remove_edge(1,1)

# Type of the graph
print('Is it a tree? ', nx.is_tree(G))
print('Is it directed? ', nx.is_directed(G))
print('Is it rooted? ', nx.is_arborescence(G)) # to check if it is rooted
print('Is it a DAG? ', nx.is_directed_acyclic_graph(G))

# The number of nodes
print('The number of nodes: ', G.number_of_nodes())
\end{minted}
\item
Using the code above, we received that the NCBI taxonomy is a rooted tree and a directed acyclic graph, and it is NOT a directed graph with cycles (no cycles in the graph).

\item 
The NCBI taxonomy has 2442791 nodes.
\item 
\setminted{autogobble=true, breaklines=true}
\begin{minted}{python}
valid_ranks = ["kingdom", "phylum", "class", "order", "family", "genus", "species"]

# Nodes to remove
not_nodes = [node for node, attrs in G.nodes(data=True) if not attrs['rank'] in valid_ranks]

restricted_taxonomy = G.copy()
restricted_taxonomy.add_edge(1,1)

def delete_nodes_preserve_neighbors(graph, nodes_to_delete):
    for node in nodes_to_delete:
        list_succes = list(graph.successors(node)) 
        list_pred = list(graph.predecessors(node))
        if node != 1:
            graph.remove_node(node)
        for source in list_pred:
            for target in list_succes:
                if source != target and not graph.has_edge(source, target):
                    graph.add_edge(source, target)

# We keep the node 1 (root) in order to have a structure of the tree 

delete_nodes_preserve_neighbors(restricted_taxonomy, not_nodes)
restricted_taxonomy.remove_edge(1,1)

print('Is it a tree? ', nx.is_tree(restricted_taxonomy))

# The number of nodes
print('The number of nodes in the restricted version: ', restricted_taxonomy.number_of_nodes())
print('The number of edges in the restricted version: ', restricted_taxonomy.number_of_edges())

# Search for the top taxonomic rank in the NCBI taxonomy
print('Is it a DAG? ', nx.is_directed_acyclic_graph(restricted_taxonomy))
long_path = nx.dag_longest_path(restricted_taxonomy)
print('The rank of the top nodes is: ', restricted_taxonomy.nodes[long_path[1]]["rank"])


\end{minted}
\item 
The name of the kingdom taxonomic rank in the NCBI taxonomy is "kingdom". To explore this, we found the longest path in the tree (depth first traversal) and took the second node after the root (node 1).
 
\item There are 2112008 nodes (including the root 1, without the root 2112007) nodes in the NCBI taxonomy, once restricted to the seven standard taxonomic ranks.

\end {enumerate}
\end{document}
